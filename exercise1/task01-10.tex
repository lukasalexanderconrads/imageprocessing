
\subsection*{task 1.10 \\[1ex] getting used to universal functions}

Again ---without using \keyword{for} loops--- implement a method that turns a given intensity image function $f[x, y]$ into another function $g[x, y]$ where
\begin{equation*}
g \bigl[ x, y \bigr] = \cos \Bigl( f \bigl[ x, y \bigr] \cdot \nu \cdot \tfrac{2\,\pi}{255} \Bigr) \cdot 127.5 + 127.5
\end{equation*}

\vspace{1cm}
Choose $\nu \in \{ 0.5, 1.0, 1.5, 2.0 \}$, apply your method to \texttt{portrait.png}, and enter your resulting images here
%%%%%
%%%%%
%%%%% enter your results here, i.e. replace "placeholder.pdf" by the names of the image files you created
%%%%%
%%%%%
\begin{center}
\includegraphics[width=0.24\textwidth]{t10-0.5.png} \hfill
\includegraphics[width=0.24\textwidth]{t10-1.png} \hfill
\includegraphics[width=0.24\textwidth]{t10-1.5.png} \hfill
\includegraphics[width=0.24\textwidth]{t10-2.png} 
\end{center}
%%%%%
%%%%%
%%%%%
%%%%%
%%%%%



\vspace{2cm}
Also, paste your code here \\[1ex]
%%%%%
%%%%%
%%%%% enter your code into the following environment
%%%%%
%%%%%
\begin{python}
def transform(arrF, nu):
    arrG = np.cos(arrF * nu * 2 * np.pi / 255) * 127.5 + 127.5
    return arrG
\end{python}
%%%%%
%%%%%
%%%%%
%%%%%
%%%%%






