
\subsection*{task 1.1 \\[1ex] the emboss effect}

In the \texttt{Data} folder for this exercise, you will find the intensity image
\begin{quote}
    \texttt{portrait.png}
\end{quote}
Read it into a numpy array \texttt{arrF} and print the shape of this array to determine its number of rows and columns.
\color{blue} \\[1ex]
%%%%%
%%%%%
%%%%% enter your result here
%%%%%
%%%%%
enter your result here \ldots
%%%%%
%%%%%
%%%%%
%%%%%
%%%%%
\color{black}

In the lecture, we discussed the idea of ``embossing'' an image such that the resulting image resembles a copper engraving. In fact, we discussed 4 different methods to accomplish this, namely
\begin{python}[emph={embossV1,embossV2,embossV3,embossV4}]
def embossV1(arrF):
    M, N = arrF.shape
    arrG = np.zeros((M,N))

    for i in range(1,M-1):
        for j in range(1,N-1):
            arrG[i,j] = 128 + arrF[i+1,j+1] - arrF[i-1,j-1]
            arrG[i,j] = np.maximum(0, np.minimum(255, arrG[i,j]))

    return arrG


def embossV2(arrF):
    M, N = arrF.shape
    arrG = np.zeros((M,N))
    
    arrG[1:M-1,1:N-1] = 128 + arrF[2:,2:] - arrF[:-2,:-2]
    arrG = np.maximum(0, np.minimum(255, arrG))

    return arrG


def embossV3(arrF):
    mask = np.array([[-1, 0,  0],
                     [ 0, 0,  0],
                     [ 0, 0, +1]])

    arrG = 128 + img.correlate(arrF, mask, mode='reflect')
    arrG = np.maximum(0, np.minimum(255, arrG))

    return arrG
    
  
def embossV4(arrF):
    arrG = 128 + arrF[2:,2:] - arrF[:-2,:-2]
    arrG[arrG<  0] =   0
    arrG[arrG>255] = 255

    return arrG
\end{python}

\newpage

Apply each of the above methods to \texttt{arrF} to produce a corresponding array \texttt{arrG} and write each of your results as a PNG image.

Does the result you obtain from \keyword{embossV4} differ from the results produced by the other methods? It should! Discuss the difference!
\color{blue} \\[1ex]
%%%%%
%%%%%
%%%%% enter your discussion here
%%%%%
%%%%%
enter your discussion here \ldots
%%%%%
%%%%%
%%%%%
%%%%%
%%%%%
\color{black}

