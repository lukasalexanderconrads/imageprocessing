
\subsection*{task 2.5 \\[1ex] why do we always work with small images ?}

Some of you may wonder why the example images we consider in these exercises are always rather small (say of resultion $256 \times 256$ pixels) \ldots To make a long story short, this is in order not to tax you patience. We already saw that large images take longer to process. Just for the fun it, let us therefore test your patience and pixelize a larger color image.  

In the \texttt{Data} folder for this exercise, you will find the image
\begin{quote}
    \texttt{bauckhage.jpg}
\end{quote}
This image has a resolution of $4068 \times 2712$ pixels and is therefore still of moderate size given present day standards. Read this color image into a \emph{numpy} array \texttt{arrF}, run \keyword{medianSuperPixel} on each of its color layers, and write your result as a PNG image. Experiment with different tile sizes $m \times n$ and paste one of your results here \\[1cm]
%%%%%
%%%%%
%%%%% enter your result here, i.e. replace "placeholder.pdf" by the name of your resulting image file
%%%%%
%%%%%
\begin{center}
\includegraphics[height=0.4\textwidth,width=0.6\textwidth]{t5.png} 
\end{center}
%%%%%
%%%%%
%%%%%
%%%%%
%%%%%





